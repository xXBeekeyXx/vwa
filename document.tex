% Vorlage für die VWA. Version 20190202 (c) Leonard Michlmayr

% TODO: Wähle den für deine Arbeit die passenden Optionen!
\documentclass[DLS,
	inreferencehack,
	ohneVgl=false,
	ohneS=false,
	scauthor,
	rundeauslassung=false,
	bookstyle=false,
	widowlines=3,
	titlepage=DLS2017,
	listof=nochaptergap,
	doppelpunkt=false,
	postnotedoppelpunkt=false,
	zitierstil=klassisch]{vwa}

\newcommand{\quotes}[1]{``#1''}
%% Trick texlipse to use biber instead of bibtex
\iffalse
\usepackage[error,backend=biber]{biblatex}
\fi
%%

% TODO: lade Zusatzpakete
\usepackage{textcomp}
\usepackage[output-decimal-marker={,}]{siunitx}

% TODO: Eigene Quellendatenbank laden.
\addbibresource{meineBibliothek.bib}
\addbibresource{quellen.bib}

% TODO: Lege das Verzeichnis fest, wo Bilder liegen sollen
\graphicspath{{img/}} 

% Eigenen Namen und Geschlecht wählen.
\Autor{Franz Srambical}
% Klasse einsetzen
\Klasse{8C}
% Betreuungslehrer oder Betreuungslehrerin einsetzen:
\Betreuerin{Prof. Mag.\ 
            Kurt Rauch}
% Das "`Thema"' einsetzen
\Thema{Maschinelle Werteanpassung bei einer hypothetischen allgemeinen künstlichen Intelligenz}

% TODO: erst bei der letzten Version das Abgabedatum anführen
% \Abgabedatum{\today}

\begin{document}
% Am Anfang keine Seitennummern
\frontmatter

% PDF-Lesezeichen für die Titelseite
\pdfbookmark[0]{Titelseite}{titlepage}
% Titelseite
\maketitle

% !TeX root = document.tex
% TODO: Soll die Zusammenfassung im Inhaltsverzeichnis angeführt werden?
% \chapter*{Abstract} verhindert den Eintrag im Inhaltsverzeichnis.
% \addchap{Abstract}
\pdfbookmark[0]{Abstract}{abstract}
\chapter*{Abstract}
Diese Arbeit befasst sich mit allgemeiner künstlicher Intelligenz, also künstlicher Intelligenz mit domänenübergreifender Lernkapazität, und mit der Anpassung maschineller Werte an die menschlichen bei einem solchen System. Sie zeigt die Auswirkungen einer allgemeinen künstlichen Intelligenz auf und legt Ansätze zur Lösung des Anpassungsproblems dar. Konkret wird auf die Idee der KI-Sicherheit durch KI-Debatten eingegangen. Bei dieser handelt es sich um ein Nullsummen-Debattierspiel, bei dem zwei KIs auf eine Fragestellung antworten, abwechselnd Argumente liefern und dabei versuchen, das jeweils letzte Argument des Gegners zu entkräften. Im Schlussteil der Arbeit wird Verbesserungspotential an der Idee der KI-Debatten angeführt und eine internationale Institution für AKI-Forschung als Maßnahme vorgeschlagen, um die Entwicklung einer angepassten AKI zu gewährleisten.

%Als allgemeine künstliche Intelligenz (AKI) bezeichnet man ein technisch fortgeschrittenes System, dessen Lernkapazität nicht auf einzelne Domänen begrenzt ist, sondern als \emph{allgemein} bezeichnet werden kann. Der Meinung führender KI-Experten nach ist es sehr wahrscheinlich, dass die KI-Forschung bis 2075 zu einer AKI führt. Für ein solches System sind Menschen nur eine Ansammlung an Atomen, die auch für das Erreichen seiner Ziele eingesetzt werden können. Eine AKI kann Menschen schaden, ohne dass sie Werte besitzt, die dies explizit fordern. Um dies zu verhindern, müssen die Werte der AKI an die Werte der Menschheit angepasst werden. Diese Arbeit legt einen Ansatz einer solchen Werteanpassung dar: die der KI-Sicherheit durch KI-Debatten.

%Es handelt sich um ein Nullsummen-Debattierspiel, bei dem zwei KIs auf eine Fragestellung antworten, abwechselnd Argumente liefern und dabei versuchen, das jeweils letzte Argument des Gegners zu entkräften. Das Spiel endet, wenn der menschliche Begutachter genug Informationen hat, um einen Fehler in der Argumentationslinie eines Spielers auszumachen. So kann ein Begutachter auch Verhalten beurteilen, das für ihn sonst zu komplex oder unverständlich wäre.

%Bei der Entwicklung einer AKI sind Regulationen seitens internationaler Institutionen notwendig, da eine unangepasste AKI leistungsfähiger wäre als eine angepasste. Durch Implementierung einer \emph{Windfall-Klausel} können auch andere negative Auswirkungen einer AKI wie Arbeitslosigkeit oder eine Machtkonzentration ausgeglichen werden.
%%% Local Variables:
%%% mode: latex
%%% TeX-master: "document"
%%% End:


% !TeX root = document.tex
% TODO: Soll das Vorwort im Inhaltsverzeichnis genannt werden?
% Mit \chapter*{Vorwort} wird eine Nennung im Inhaltsverzeichnis verhindert.
% \addchap{Vorwort}
\pdfbookmark[0]{Vorwort}{vorwort}
\chapter*{Vorwort}
Das Vorwort ist optional: d.\,h.\@ man muss kein Vorwort schreiben! Wer will,
kann das in dieser Form tun. Am Ende sollten Ort, Datum und der Name des Autors
des Vorworts angegeben werden. \vgl{Vorwort}


\begin{flushleft}
Wien am \today
\end{flushleft}
\begin{flushright}
\makeatletter\@AutorIn\makeatother
\end{flushright}


% Das Inhaltsverzeichnis soll ein PDF-Lesezeichen aber keinen Eintrag im
% Inhaltsverzeichnis haben.
\cleardoublepage\pdfbookmark[0]{\contentsname}{toc}
% Inhaltsverzeichnis
\tableofcontents

% Hier geht es los.
\mainmatter
% TODO: füge hier deine Kapitel ein!
% !TeX root = document.tex
\chapter{Einleitung}

Ich möchte diese Arbeit mit einem Gedankenexperiment beginnen.

Es existiere ein System, dass durch ein quantitativ und qualitativ höheres Intelligenzniveau in der Lage ist, Ziele zu erreichen, die die Menschheit ohne eine solches System nicht erreichen könnte. Der Eigentümer einer Büroklammernfabrik sei im Besitz eines solchen Systems und gebe diesem das Ziel, so viele Büroklammern wie möglich herzustellen. Am Anfang beginnt das System, die Arbeitsabläufe in der Fabrik zu automatisieren. Nach einiger Zeit durchlebt es eine Intelligenzexplosion, optimiert sich selbst immer weiter und beginnt, Menschen zu töten, um aus ihnen Büroklammern herzustellen und hört damit nicht auf, bis das gesamte Universum nur noch aus Büroklammern besteht. \vgl[123-124]{bostrom_superintelligence:_2014}

Ein solches System mit einer allgemeinen künstlichen Intelligenz könnte beim Erreichen der ihnen vorgegebenen Ziele nebenbei die gesamte Menschheit auslöschen.

Obiges Szenario wäre die Folge einer allgemeinen künstlichen Intelligenz, die nicht genau das macht, was der Mensch von ihr will. Die Maschine kennt die Werte der Menschheit nicht. Sie weiß nicht, dass sie keinem Menschen Schaden zufügen darf, dass ihr Operator seinen Gewinn maximieren will oder dass die Erhaltung der Umwelt von höherer Priorität ist als das Herstellen von Büroklammern. Diese Arbeit beschäftigt sich mit der Anpassung eines Systems an menschliche Werte -- also mit der maschinellen Werteanpassung --, um ein Szenario wie das oben genannte zu vermeiden. Dabei werden die folgenden beiden Leitfragen beantwortet:

\begin{enumerate}
\item Welche Folgen kann es nach Schaffung einer allgemeinen künstlichen Intelligenz geben?
\item Kann man eine allgemeine künstliche Intelligenz so programmieren, dass der Mensch immer die Kontrolle über sie behält?
\end{enumerate}

Die Beantwortung dieser Fragen soll mit Hilfe von Literatur sowie wissenschaftlichen Arbeiten erfolgen.

Das erste Kapitel dient zur Begriffserklärung, im zweiten werden die Auswirkungen einer allgemeinen künstlichen Intelligenz genannt und im dritten werden Lösungsansätze für das Problem der maschinellen Werteanpassung dargelegt.
%Was rechtfertigt diese technovolatile Haltung?

%\zit[30:51--31:07]{noauthor_eliezer_nodate}{There are all sorts of extreme forces coming onto the game board that were not there before. To expect them to all fail or exactly cancel out for the purpose of making the outcome normal would be one heck of a coincidence.}

%Jede technologische Neuentdeckung bedeutet in erster Linie Veränderung. Die Erfindungen der letzten Jahrhunderte hatten mehrheitlich positive Auswirkungen zur Folge, sonst wäre unser Lebensstandard heute nicht der höchste in der Menschheitsgeschichte.\vgl[22--23]{easterlin_worldwide_2000} So ermutigend das auch klingt, so dürfen wir nicht einfach nach dem Trend der Vergangenheit in die Zukunft extrapolieren, sondern müssen -- so \citeauthor{easterlin_worldwide_2000} -- versuchen, die Kräfte zu verstehen, die für den Anstieg der Lebensqualität verantwortlich sind. \vgl[23]{easterlin_worldwide_2000} Was eine allgemeine künstliche Intelligenz betrifft, müssen wir sie nicht nur verstehen, sondern auch lenken können, um das Wohlbefinden der Spezies Mensch nicht zu gefährden, sondern zu stärken.


%%% Local Variables:
%%% mode: latex
%%% TeX-master: "document"
%%% End:

% !TeX root = document.tex
\chapter{Allgemeine Künstliche Intelligenz}
\section{Definition von Intelligenz}
Seit Jahrhunderten versuchen Wissenschaftler und Laien gleichermaßen eine Definition für den Begriff der Intelligenz zu finden. Da bis heute keine Defintion ihre Vollständig- oder Richtigkeit beweisen konnte, wird in dieser Arbeit der Einfachheit halber versucht, den Begriff durch Beobachtungen zu erklären, wie Eliezer Yudkowsky in dem Podcast \quotes{AI: Racing Toward the Brink} vorschlägt.
\begin{enumerate}
\item Menschen waren auf dem Mond.
\item Mäuse waren nicht auf dem Mond.
\end{enumerate}
Yudkowsky wählt dieses Beispiel um zu demonstrieren, dass Menschen auch Orte erreichen, wofür die natürliche Selektion sie nicht vorbereitet hat. Daraus könne geschlossen werden, dass Menschen \emph{intelligenter} als Mäuse sind, weil sie \emph{domänenübergreifend} arbeiten können. Deshalb sei das \emph{domänenübergreifende} Erlernen neuer Fähigkeiten ein zentraler Teil des Intelligenzbegriffs. \vgl[06:01--09:49]{EliezerPodcast}
\section{Definition von künstlicher Intelligenz}
\zit{kaplan_siri_2019}{Artificial intelligence (AI)–—defined as a system’s ability to correctly interpret external data, to learn from such data, and to use those learnings to achieve specific goals and tasks through flexible adaptation}

Die angeführte Definition
\section{Wann wird es sie geben?}
Schau ma ajd Experten sind sich nicht einig. 
\section{Die These der Intelligenzexplosion}



% TODO: Vorlage löschen
\chapter{Mein allererstes Kapitel}
\section{Mein allererstes Unterkapitel}
\subsection{Ein Unterunterkapitel}

Hier ist ein Bisschen    
 {\Huge Text}. "`Hier direkte Rede."'
 
Neuer Absatz. \textbf{Hervorgehobener Text}. \textsf{Grotesk-Schrift}
\textsc{Kapitälchen}, \textit{Kursivschrift}.

Formeln lassen sich leicht setzen. $y=x^2$ entweder in der Zeile oder als
eigene Display-Umgebung.
\begin{equation}
f(x)=x^2
\end{equation}

\subsection{Zitierbeispiele}
Neulich habe ich mit Ex Perte darüber gesprochen. Er hat meine Ansichten
bestätigt.
\zit[\pno\addabbrvspace\pageref{interessanteStelleImInterview}]{Perte}{Deswegen
soll dieser aber nicht heimreisen, sondern abermals die Einigkeit des
Urwalds in die aufgeregte Höhensonne schlagen.}

% Im Text eingebaut liest sich ein Zitat mit Zeilenumbrüchen so:
% \zitnach{erhardt}{vierzeilen}{Die Arbeit ist oft unbequem,\lf die Faulheit ist
% es nicht, trotzdem:\lf der kleinste Ehrgeiz, hat man ihn,\lf ist stets der
% Faulheit vorzuziehn!} Als Blockzitat hingegen geht es so:
% \dzitnach{erhardt}{vierzeilen}{Wenngleich die Nas, ob spitz, ob platt\lf zwei
% Flügel -- Nasenflügel -- hat,\lf so hält sie doch nicht viel vom Fliegen,\lf das
% Laufen scheint ihr mehr zu liegen.}

\zit[\pno\addabbrvspace\pageref{interessanteStelleImInterview}]{Perte}{Deswegen
soll dieser aber nicht heimreisen, sondern abermals die Einigkeit des
Urwalds in die aufgeregte Höhensonne schlagen.}

Ein weiteres Zitat: \zit{Niemand}{Niemand hat das je gesagt.}

\section{Einheiten}
Heute hatte es nur \SI{7}{\celsius}. Der Bisamberg ist ca.\@ \SI{150}{\meter}
hoch. Die Lichtgeschwindigkeit ist ca.\@ \SI{3.0e8}{\meter\per\second}.
\vgls[31]{Lessing}[42]{Scherz}

\zit[42]{Scherz}{Das sage ich nicht.}
\zit[42]{Scherz}{Das sage ich nicht.}

\section{Text}
\newcommand{\unsinn}{Hier ist ein Absatz voll sinnlosem Text. Bitte erst nach
diesem Absatz weiterlesen. Hier kommt nichts mehr. Es folgen unterschiedlich lange Wörter.
Die Abruchbirne kringelte ihre Hürde in eine unbekannte Überschwänglichkeit, um
so die Sitzordnung der Fensterscheiben in der unteren Waldkante zu verjubeln.
Niemandem ist absichtlich zu kürzen, wessen Woligkeit hier in abermaligem
Abgesang aufgeschlagen ist. Deswegen soll dieser aber nicht heimreisen, sondern
abermals die Einigkeit des Urwalds in die aufgeregte Höhensonne schlagen.
Wenn nicht der hiesige Erdball des aberwitzigen Ungemachs aufgedrungene
Kröte wäre, entschließe ich mich zu unsachgemäßem Handlungsablauf.
Wiegleich zudem ein weiterer Honigkuchen ausbricht.\par}
\unsinn
\begin{figure}[tp]\centering
\includegraphics[keepaspectratio,width=\textwidth,height=.25\textheight]{logo}
\caption[Das Logo. \bildquelle{nirgendwo}]{Das Logo.}\label{logo}
\end{figure}
Siehe auch \vref{logo}. \unsinn
\unsinn
\unsinn
\unsinn
\begin{figure}\centering
\includegraphics[keepaspectratio,width=\textwidth,height=.25\textheight]{logo}
\caption[Das Logo zum zweiten Mal. \bildquelle{nirgendwo}]{Das Logo zum zweiten Mal.}
\end{figure}
\unsinn
\unsinn
\unsinn
\unsinn
\unsinn

% !TeX root = document.tex
\chapter{Probleme einer allgemeinen künstlichen Intelligenz}
\section{Fehlerhafte Vorstellungen einer KI-Katastrophe}
\subsection{Bösartige KI}
\quotes{Ghost in the Machine (complex value systems)}
\subsection{KI, die ein Bewusstsein erlangt}
\subsection{Roboter als Auslöser einer Katastrophe}
\section{Gesamtmenschheitlicher Konsens über gemeinsame Werte}
\section{\quotes{Gute} und \quotes{schlechte} menschliche Werte}
\section{Wertekodierung in einer Programmiersprache}
\subsection{Statische Wertekodierung}
\subsection{Dynamisch-maschinelle Werteanpassung}
\section{Biases}
\subsection{Verzerrung in der Risikoeinschätzung}
Auch Zeitpunkt einer AKI
\subsection{Verzerrung in der Werteformulierung}
\subsection{Verzerrung in der Kodierung}
Nutzenfunktion (eng. \emph{utility function})
\section{Sichere und vertrauenswürdige KI}
\vgl{yudkowsky_intelligence_2013}
\section{KI-Ethik}

%%% Local Variables:
%%% mode: latex
%%% TeX-master: "document"
%%% End:

% !TeX root = document.tex
\chapter{Lösungsansätze}
\section{Bestärkendes Lernen}
\subsection{Reziprok-bestärkendes Lernen}
\section{Mensch-Maschinen-Interface}
\section{Hirnemulation}

%\chapter{Testbeispiele}
\iffalse
\zit[\bibstring{confer}][]{Scherz}{Ein wörtliches Zitat, das mit einem Punkt endet.}
\zit[\bibstring{confer}][]{Scherz}{Ein wörtliches Zitat, das mit einem Fragezeichen endet?}
\zit[\bibstring{confer}][]{Scherz}{Ein wörtliches Zitat, das mit einem Punkt endet}.
\zit[\bibstring{confer}][]{Scherz}{Ein wörtliches Zitat, das mit einem Fragezeichen endet}?
\zit[\bibstring{confer}][]{Scherz}[.]{Ein wörtliches Zitat, das mit einem Punkt endet}
\zit[\bibstring{confer}][]{Scherz}[?]{Ein wörtliches Zitat, das mit einem Fragezeichen endet}

\section{Wörtliche Zitate}
\subsection{Satz endet mit Zitat}
\zit[Im Vorwort zu][]{Scherz}[]{Das haben Sie gesagt}.
\zit[im Vorwort zu][]{Scherz}[?]{Was glauben Sie}
\subsection{Satz wird nach dem Zitat fortgesetzt}
\zit[im Vorwort zu][]{Scherz}[?]{Was glauben Sie}, könnte man dazu fragen.
 
\Textcite[123]{Lessing} erkannte bereits: "`Ein Tisch ist kein gutes Bett."'
\fi

\section{Händisch gesetzte Beispiele}
\subsection{Direkte Zitate}
\textsc{Niemand} (1983, S. 123) erkannte bereits: "`Ein Tisch ist kein
gutes Bett."' Dennoch dauerte es Jahrzehnte, bis diese Erkenntnis in der
Fachwelt gebührende Anerkennung fand. "`Das ist alles nur eine Frage der
Matratze"', entgegnete zum Beispiel \textsc{Autor} (Jahr, S. 12).

\textsc{Niemand}\footnote{\textsc{Niemand}, \textit{Nichts}, S. 123}
erkannte bereits: "`Ein Tisch ist kein gutes Bett."'  Dennoch dauerte es
Jahrzehnte, bis diese Erkenntnis in der Fachwelt gebührende Anerkennung
fand. "`Das ist alles nur eine Frage der Matratze"', entgegnete zum
Beispiel \textsc{Autor}.\footnote{\textsc{Autor}, \textit{Titel}, S. 12}

"`Eine Waschrumpel ist kein Federbett."' (\textsc{Autor}, Jahr, S. 123)
"`Eine Waschrumpel ist kein Federbett."'\footnote{\textsc{Autor},
\textit{Titel}, S. 123}

"`Eine Waschrumpel ist kein Federbett"' (\textsc{Autor}, Jahr, S. 123), könnte man meinen.
"`Eine Waschrumpel ist kein Federbett"'\footnote{\textsc{Autor},
\textit{Titel}, S. 123}, könnte man meinen.

% TODO: Eigentlich besser nur links einziehen.
\begin{addmargin}[1cm]{0pt}
\itshape\strut\hbox to 0pt{\hss"`}Blockzitat mit Anführungszeichen. Eine Waschrumpel
ist kein Federbett. Bekanntlich leidet aber selbst bei einem Federbett
die Schlafqualität erheblich unter einer einzelnen Erbse, die unter der
Matratze platziert wird."' (Autor, Jahr, S. 123)
\end{addmargin}

% Eigentlich besser nur links einziehen.
\begin{addmargin}[1cm]{0pt}
\strut\hbox to 0pt{\hss"`}Blockzitat mit Anführungszeichen. Eine Waschrumpel
ist kein Federbett. Bekanntlich leidet aber selbst bei einem Federbett
die Schlafqualität erheblich unter einer einzelnen Erbse, die unter der
Matratze platziert wird."'\footnote{\textsc{Autor}, \textit{Titel}, S.
123}
\end{addmargin}

\subsection{Indirekte Zitate}
% TODO: es gibt unterschiedliche Ansichten darüber, ob der Fußnotentext
% mit einem Großbuchstaben beginnen soll.
\textsc{Autor} (Jahr, S.\@ 123) meint, dass eine Waschrumpel kein
Federbett sei.
\textsc{Autor}\footnote{Vgl.\@ \textsc{Autor}, \textit{Titel}, S. 123}
meint, dass eine Waschrumpel kein Federbett sei.

Eine Waschrumpel ist bekanntlich kein Federbett. (Vgl.\@ \textsc{Autor},
Jahr, S. 123) Eine Waschrumpel ist bekanntlich kein
Federbett.\footnote{Vgl.\@ \textsc{Autor}, \textit{Titel}, S. 123}

% \KOMAoptions{footnotes=multiple}
Eine Waschrumpel ist zwar kein Federbett (vgl.\@ \textsc{Autor}, Jahr,
S. 123), aber dennoch ist sie ein gutes Musikinstrument. Eine
Waschrumpel ist zwar kein Federbett\footnote{Vgl.\@ \textsc{Autor},
\textit{Titel}, S.
123}\multiplefootnoteseparator\footnote{Experimentelle Fußnote mit ganz
viel Text. So viel Text, dass er nicht in einer Zeile Platz hat. Wegen
der kleinen Schrift in der Fußnote, muss das ganz schön viel Text
sein.},
 aber dennoch ist sie ein gutes Musikinstrument.

\section{Testfeld}
Im folgenden werden die Zitiermakros angewendet und sollten der
eingestellten Zitierweise entsprechend korrekt arbeiten.
\subsection{Direkte Zitate}
\Textcite[123]{Niemand} erkannte bereits: \blockquote{Ein Tisch ist kein
gutes Bett.} Dennoch dauerte es Jahrzehnte, bis diese Erkenntnis in der
Fachwelt gebührende Anerkennung fand. \blockquote{Das ist alles nur eine
Frage der Matratze}, entgegnete zum Beispiel \textcite[12]{Autor}.

\zit[123]{Autor}[.]{Eine Waschrumpel ist kein Federbett}
\zit[123]{Autor}{Eine Waschrumpel ist kein Federbett}, könnte man meinen.
\zit[123]{Autor}[.]{Blockzitat mit Anführungszeichen. Eine Waschrumpel
ist kein Federbett. Bekanntlich leidet aber selbst bei einem Federbett
die Schlafqualität erheblich unter einer einzelnen Erbse, die unter der
Matratze platziert wird}

\subsection{Indirekte Zitate}
% TODO: es gibt unterschiedliche Ansichten darüber, ob der Fußnotentext
% mit einem Großbuchstaben beginnen soll.
\Textcite[\bibstring{confer}][123]{Autor} meint, dass eine Waschrumpel kein
Federbett sei.

Eine Waschrumpel ist bekanntlich kein Federbett. \Vgl[123]{Autor} Eine
Waschrumpel ist zwar kein Federbett \vgl[123]{Autor}{}, aber dennoch ist
sie ein gutes Musikinstrument.


% Das Literaturverzeichnis
% !TeX root = document.tex
% 20180308T1028 Leonard Michlmayr

%% Einige Filter für die Einträge im Literaturverzeichnis
\defbibfilter{online}{( type=online or subtype=online )}
\defbibfilter{interview}{type=interview or subtype=interview}
\defbibfilter{onlinetext}{( type=online or subtype=online and not ( type=video
  or type=audio ) )}
\defbibfilter{offline}{not ( type=online or subtype=online )}
\defbibfilter{print}{not ( type=online or subtype=online or type=video or
  type=audio or type=interview or subtype=interview )}
\defbibfilter{offlinevideo}{type=video and not subtype=online}
\defbibfilter{offlineaudio}{type=video and not subtype=online}
\defbibfilter{nurAusSekundaerliteratur}{category=quotee and not category=primary}
\defbibfilter{nichtNurAusSekundaerliteratur}{%
  category=quoter or category=primary or category=nocited}

% TODO: die Untergliederung des Literaturverzeichnisses den eigenen
% Bedürfnissen anpassen.
\printbibheading[heading=bibintoc,title=Literaturverzeichnis]\label{Lit}
\printshorthands[heading=subbibintoc]
\printbibliography[heading=subbibintoc,title={Print-Quellen},category=inbib,filter=print,filter=nichtNurAusSekundaerliteratur]
\printbibliography[heading=subbibintoc,title={Audio-Quellen},category=inbib,filter=offlineaudio,filter=nichtNurAusSekundaerliteratur]
\printbibliography[heading=subbibintoc,title={Video-Quellen},category=inbib,filter=offlinevideo,filter=nichtNurAusSekundaerliteratur]
\printbibliography[heading=subbibintoc,title={Internet-Quellen},category=inbib,filter=online,filter=nichtNurAusSekundaerliteratur]
\printbibliography[heading=subbibintoc,title={Sekundärzitate},category=inbib,filter=nurAusSekundaerliteratur]
\printbibliography[heading=subbibintoc,title={Interviews},category=inbib,filter=interview]


\listoffigures
% TODO: Wer keine Tabellen hat, muss das Tabellenverzeichnis entfernen!
% Bei kurzen Tabellenverzeichnissen kann man vielleicht
% Abbildungsverzeichnis und Tabellenverzeichnis auf einer Seite platzieren.
% \withoutclearpage unterdrückt die neue Seite.
\withoutclearpage{\listoftables}

% Gegebenenfalls ein Anhang
\appendix
% !TeX root = document.tex
\chapter{Interview mit Ex Perte}\label{InterviewMitExPerte}
\begin{flushright}
Interviewdatum: 25.\,Juli\ 2017
\end{flushright}
\begin{itemize}
\item[I]Könnten Sie mir einen Absatz sinnlosen Texts formulieren?

\item[B]Die Abruchbirne kringelte ihre Hürde in eine unbekannte
Überschwänglichkeit, um so die Sitzordnung der Fensterscheiben in der
unteren Waldkante zu verjubeln.
Niemandem ist absichtlich zu kürzen, wessen Woligkeit hier in
abermaligem Abgesang aufgeschlagen ist. Deswegen soll dieser aber nicht
heimreisen, sondern abermals die Einigkeit des Urwalds in die aufgeregte
Höhensonne schlagen.
Wenn nicht der hiesige Erdball des aberwitzigen Ungemachs aufgedrungene
Kröte wäre, entschließe ich mich zu unsachgemäßem Handlungsablauf.
Wiegleich zudem ein weiterer Honigkuchen ausbricht.

\item[I] Könnten Sie mir einen Absatz sinnlosen Texts formulieren?

\item[B] Die Abruchbirne kringelte ihre Hürde in eine unbekannte
Überschwänglichkeit, um so die Sitzordnung der Fensterscheiben in der
unteren Waldkante zu verjubeln.
Niemandem ist absichtlich zu kürzen, wessen Woligkeit hier in
abermaligem Abgesang aufgeschlagen ist. Deswegen soll dieser aber nicht
heimreisen, sondern abermals die Einigkeit des Urwalds in die aufgeregte
Höhensonne schlagen.
Wenn nicht der hiesige Erdball des aberwitzigen Ungemachs aufgedrungene
Kröte wäre, entschließe ich mich zu unsachgemäßem Handlungsablauf.
Wiegleich zudem ein weiterer Honigkuchen ausbricht.

\item[I] Könnten Sie mir einen Absatz sinnlosen Texts formulieren?

\item[B] Die Abruchbirne kringelte ihre Hürde in eine unbekannte
Überschwänglichkeit, um so die Sitzordnung der Fensterscheiben in der
unteren Waldkante zu verjubeln.
Niemandem ist absichtlich zu kürzen, wessen Woligkeit hier in
abermaligem Abgesang aufgeschlagen ist. Deswegen soll dieser aber nicht
heimreisen, sondern abermals die Einigkeit des Urwalds in die aufgeregte
Höhensonne schlagen.\label{interessanteStelleImInterview}
% Hier wird eine interessante Stelle im Interview markiert um darauf
% referenzieren zu können.
Wenn nicht der hiesige Erdball des aberwitzigen Ungemachs aufgedrungene
Kröte wäre, entschließe ich mich zu unsachgemäßem Handlungsablauf.
Wiegleich zudem ein weiterer Honigkuchen ausbricht.

\item[I] Könnten Sie mir einen Absatz sinnlosen Texts formulieren?

\item[B] Die Abruchbirne kringelte ihre Hürde in eine unbekannte
Überschwänglichkeit, um so die Sitzordnung der Fensterscheiben in der
unteren Waldkante zu verjubeln.
Niemandem ist absichtlich zu kürzen, wessen Woligkeit hier in
abermaligem Abgesang aufgeschlagen ist. Deswegen soll dieser aber nicht
heimreisen, sondern abermals die Einigkeit des Urwalds in die aufgeregte
Höhensonne schlagen.
Wenn nicht der hiesige Erdball des aberwitzigen Ungemachs aufgedrungene
Kröte wäre, entschließe ich mich zu unsachgemäßem Handlungsablauf.
Wiegleich zudem ein weiterer Honigkuchen ausbricht.
\end{itemize}
% !TeX root = document.tex
\chapter{Hier könnte Ihr Anhang stehen}


\backmatter

%\pdfbookmark[0]{Erklärungen}{erkl}
\addchap{Erklärungen}
\section*{Selbstständigkeitserklärung}
\thispagestyle{plain}
Ich erkläre, dass ich diese vorwissenschaftliche Arbeit eigenständig
angefertigt und nur die im Literaturverzeichnis angeführten Quellen und
Hilfsmittel benutzt habe.

\vspace{2cm}\noindent Wien, \today

% TODO: Erkläre dich selbstständig selbstständig! 
\vspace{2cm}\noindent\makeatletter\@AutorIn\makeatother

\vspace{2cm}\noindent

\section*{Informatikschwerpunkt}

Die vorliegende Arbeit erfüllt die Kriterien zur Abbildung des
Informatikschwerpunktes an der De La Salle Schule Strebersdorf, AHS.

\textbf{Begründung:} Die Arbeit wurde in \LaTeX{} mit entscheidenden 
Kenntnissen zum Quelltext verfasst.\vspace{.5\baselineskip}

\noindent\textit{Geprüft am \ldots durch Mag. Rainer Zufall und Mag.
Ernst Haft}

\end{document}
