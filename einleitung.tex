% !TeX root = document.tex
\chapter{Einleitung}

Ich möchte diese Arbeit mit einem Gedankenexperiment beginnen.

Es existiere ein System, dass durch ein quantitativ und qualitativ höheres Lernniveau in der Lage ist, Ziele zu erreichen, die die Menschheit ohne eine solches System nicht erreichen könnte. Der Eigentümer einer Büroklammernfabrik ist im Besitz eines solchen Systems und gibt diesem das Ziel, so viele Büroklammern wie möglich herzustellen. Am Anfang beginnt das System, die Arbeitsabläufe in der Fabrik zu automatisieren. Nach einiger Zeit durchlebt es eine Intelligenzexplosion, optimiert sich selbst immer weiter und beginnt, Menschen zu töten, um aus ihnen Büroklammern herzustellen und hört damit nicht auf, bis das gesamte Universum nur noch aus Büroklammern besteht. \vgl[123-124]{bostrom_superintelligence:_2014}

Es ist durchaus möglich, dass ein solches System mit einer allgemeinen künstlichen Intelligenz beim Erreichen der ihnen vorgegebenen Ziele nebenbei die gesamte Menschheit auslöscht.

Was rechtfertigt eine so technovolatile Haltung wie diese?

\zit[30:51--31:07]{noauthor_eliezer_nodate}{There are all sorts of extreme forces coming onto the game board that were not there before. To expect them to all fail or exactly cancel out for the purpose of making the outcome normal would be one heck of a coincidence.}

Jede technologische Neuentdeckung bedeutet in erster Linie Veränderung. Die Erfindungen der letzten Jahrhunderte hatten mehrheitlich positive Auswirkungen zur Folge, sonst wäre unser Lebensstandard heute nicht der höchste in der Menschheitsgeschichte.\vgl[22--23]{easterlin_worldwide_2000} So ermutigend das auch klingt, so dürfen wir nicht einfach nach dem Trend der Vergangenheit in die Zukunft extrapolieren, sondern müssen - so Richard A. Easterlin - versuchen, die Kräfte zu verstehen, die für den Anstieg der Lebensqualität verantwortlich sind. \vgl[23]{easterlin_worldwide_2000} Was eine allgemeine künstliche Intelligenz betrifft, müssen wir sie nicht nur verstehen, sondern auch lenken können, um das Wohlbefinden der Spezies Mensch nicht zu gefährden, sondern zu bestärken.

Büroklammern Ziel Beispiel
Werte formulieren die keinen schaden anrichten (Beispiel finden)
VERZERRUNG BIASES übersetzung

