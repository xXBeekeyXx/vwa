% !TeX root = document.tex
\chapter{Allgemeine Künstliche Intelligenz}
\section{Definition von Intelligenz}
Seit Jahrhunderten versuchen Wissenschaftler und Laien gleichermaßen eine Definition für den Begriff der Intelligenz zu finden. Da bis heute keine Defintion ihre Vollständig- oder Richtigkeit beweisen konnte, wird in dieser Arbeit der Einfachheit halber versucht, den Begriff durch Beobachtungen zu erklären, wie Eliezer Yudkowsky in dem Podcast \quotes{AI: Racing Toward the Brink} vorschlägt.
\begin{enumerate}
\item Menschen waren auf dem Mond.
\item Mäuse waren nicht auf dem Mond.
\end{enumerate}
Yudkowsky wählt dieses Beispiel um zu demonstrieren, dass Menschen auch Orte erreichen, wofür die natürliche Selektion sie nicht vorbereitet hat. Daraus könne geschlossen werden, dass Menschen \emph{intelligenter} als Mäuse sind, weil sie \emph{domänenübergreifend} arbeiten können. Deshalb sei das \emph{domänenübergreifende} Erlernen neuer Fähigkeiten ein zentraler Teil des Intelligenzbegriffs. \vgl[06:01--09:49]{EliezerPodcast}
\section{Definition von künstlicher Intelligenz}
\zit{kaplan_siri_2019}{Artificial intelligence (AI)–—defined as a system’s ability to correctly interpret external data, to learn from such data, and to use those learnings to achieve specific goals and tasks through flexible adaptation}

Die angeführte Definition
\section{Wann wird es sie geben?}
Schau ma ajd Experten sind sich nicht einig. 
\section{Die These der Intelligenzexplosion}



% TODO: Vorlage löschen
\chapter{Mein allererstes Kapitel}
\section{Mein allererstes Unterkapitel}
\subsection{Ein Unterunterkapitel}

Hier ist ein Bisschen    
 {\Huge Text}. "`Hier direkte Rede."'
 
Neuer Absatz. \textbf{Hervorgehobener Text}. \textsf{Grotesk-Schrift}
\textsc{Kapitälchen}, \textit{Kursivschrift}.

Formeln lassen sich leicht setzen. $y=x^2$ entweder in der Zeile oder als
eigene Display-Umgebung.
\begin{equation}
f(x)=x^2
\end{equation}

\subsection{Zitierbeispiele}
Neulich habe ich mit Ex Perte darüber gesprochen. Er hat meine Ansichten
bestätigt.
\zit[\pno\addabbrvspace\pageref{interessanteStelleImInterview}]{Perte}{Deswegen
soll dieser aber nicht heimreisen, sondern abermals die Einigkeit des
Urwalds in die aufgeregte Höhensonne schlagen.}

% Im Text eingebaut liest sich ein Zitat mit Zeilenumbrüchen so:
% \zitnach{erhardt}{vierzeilen}{Die Arbeit ist oft unbequem,\lf die Faulheit ist
% es nicht, trotzdem:\lf der kleinste Ehrgeiz, hat man ihn,\lf ist stets der
% Faulheit vorzuziehn!} Als Blockzitat hingegen geht es so:
% \dzitnach{erhardt}{vierzeilen}{Wenngleich die Nas, ob spitz, ob platt\lf zwei
% Flügel -- Nasenflügel -- hat,\lf so hält sie doch nicht viel vom Fliegen,\lf das
% Laufen scheint ihr mehr zu liegen.}

\zit[\pno\addabbrvspace\pageref{interessanteStelleImInterview}]{Perte}{Deswegen
soll dieser aber nicht heimreisen, sondern abermals die Einigkeit des
Urwalds in die aufgeregte Höhensonne schlagen.}

Ein weiteres Zitat: \zit{Niemand}{Niemand hat das je gesagt.}

\section{Einheiten}
Heute hatte es nur \SI{7}{\celsius}. Der Bisamberg ist ca.\@ \SI{150}{\meter}
hoch. Die Lichtgeschwindigkeit ist ca.\@ \SI{3.0e8}{\meter\per\second}.
\vgls[31]{Lessing}[42]{Scherz}

\zit[42]{Scherz}{Das sage ich nicht.}
\zit[42]{Scherz}{Das sage ich nicht.}

\section{Text}
\newcommand{\unsinn}{Hier ist ein Absatz voll sinnlosem Text. Bitte erst nach
diesem Absatz weiterlesen. Hier kommt nichts mehr. Es folgen unterschiedlich lange Wörter.
Die Abruchbirne kringelte ihre Hürde in eine unbekannte Überschwänglichkeit, um
so die Sitzordnung der Fensterscheiben in der unteren Waldkante zu verjubeln.
Niemandem ist absichtlich zu kürzen, wessen Woligkeit hier in abermaligem
Abgesang aufgeschlagen ist. Deswegen soll dieser aber nicht heimreisen, sondern
abermals die Einigkeit des Urwalds in die aufgeregte Höhensonne schlagen.
Wenn nicht der hiesige Erdball des aberwitzigen Ungemachs aufgedrungene
Kröte wäre, entschließe ich mich zu unsachgemäßem Handlungsablauf.
Wiegleich zudem ein weiterer Honigkuchen ausbricht.\par}
\unsinn
\begin{figure}[tp]\centering
\includegraphics[keepaspectratio,width=\textwidth,height=.25\textheight]{logo}
\caption[Das Logo. \bildquelle{nirgendwo}]{Das Logo.}\label{logo}
\end{figure}
Siehe auch \vref{logo}. \unsinn
\unsinn
\unsinn
\unsinn
\begin{figure}\centering
\includegraphics[keepaspectratio,width=\textwidth,height=.25\textheight]{logo}
\caption[Das Logo zum zweiten Mal. \bildquelle{nirgendwo}]{Das Logo zum zweiten Mal.}
\end{figure}
\unsinn
\unsinn
\unsinn
\unsinn
\unsinn
