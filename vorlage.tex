% TODO: Vorlage löschen
\chapter{Mein allererstes Kapitel}
\section{Mein allererstes Unterkapitel}
\subsection{Ein Unterunterkapitel}

Hier ist ein Bisschen    
 {\Huge Text}. "`Hier direkte Rede."'
 
Neuer Absatz. \textbf{Hervorgehobener Text}. \textsf{Grotesk-Schrift}
\textsc{Kapitälchen}, \textit{Kursivschrift}.

Formeln lassen sich leicht setzen. $y=x^2$ entweder in der Zeile oder als
eigene Display-Umgebung.
\begin{equation}
f(x)=x^2
\end{equation}

\subsection{Zitierbeispiele}
Neulich habe ich mit Ex Perte darüber gesprochen. Er hat meine Ansichten
bestätigt.
\zit[\pno\addabbrvspace\pageref{interessanteStelleImInterview}]{Perte}{Deswegen
soll dieser aber nicht heimreisen, sondern abermals die Einigkeit des
Urwalds in die aufgeregte Höhensonne schlagen.}

% Im Text eingebaut liest sich ein Zitat mit Zeilenumbrüchen so:
% \zitnach{erhardt}{vierzeilen}{Die Arbeit ist oft unbequem,\lf die Faulheit ist
% es nicht, trotzdem:\lf der kleinste Ehrgeiz, hat man ihn,\lf ist stets der
% Faulheit vorzuziehn!} Als Blockzitat hingegen geht es so:
% \dzitnach{erhardt}{vierzeilen}{Wenngleich die Nas, ob spitz, ob platt\lf zwei
% Flügel -- Nasenflügel -- hat,\lf so hält sie doch nicht viel vom Fliegen,\lf das
% Laufen scheint ihr mehr zu liegen.}

\zit[\pno\addabbrvspace\pageref{interessanteStelleImInterview}]{Perte}{Deswegen
soll dieser aber nicht heimreisen, sondern abermals die Einigkeit des
Urwalds in die aufgeregte Höhensonne schlagen.}

Ein weiteres Zitat: \zit{Niemand}{Niemand hat das je gesagt.}

\section{Einheiten}
Heute hatte es nur \SI{7}{\celsius}. Der Bisamberg ist ca.\@ \SI{150}{\meter}
hoch. Die Lichtgeschwindigkeit ist ca.\@ \SI{3.0e8}{\meter\per\second}.
\vgls[31]{Lessing}[42]{Scherz}

\zit[42]{Scherz}{Das sage ich nicht.}
\zit[42]{Scherz}{Das sage ich nicht.}

\section{Text}
\newcommand{\unsinn}{Hier ist ein Absatz voll sinnlosem Text. Bitte erst nach
diesem Absatz weiterlesen. Hier kommt nichts mehr. Es folgen unterschiedlich lange Wörter.
Die Abruchbirne kringelte ihre Hürde in eine unbekannte Überschwänglichkeit, um
so die Sitzordnung der Fensterscheiben in der unteren Waldkante zu verjubeln.
Niemandem ist absichtlich zu kürzen, wessen Woligkeit hier in abermaligem
Abgesang aufgeschlagen ist. Deswegen soll dieser aber nicht heimreisen, sondern
abermals die Einigkeit des Urwalds in die aufgeregte Höhensonne schlagen.
Wenn nicht der hiesige Erdball des aberwitzigen Ungemachs aufgedrungene
Kröte wäre, entschließe ich mich zu unsachgemäßem Handlungsablauf.
Wiegleich zudem ein weiterer Honigkuchen ausbricht.\par}
\unsinn
\begin{figure}[tp]\centering
\includegraphics[keepaspectratio,width=\textwidth,height=.25\textheight]{logo}
\caption[Das Logo. \bildquelle{nirgendwo}]{Das Logo.}\label{logo}
\end{figure}
Siehe auch \vref{logo}. \unsinn
\unsinn
\unsinn
\unsinn
\begin{figure}\centering
\includegraphics[keepaspectratio,width=\textwidth,height=.25\textheight]{logo}
\caption[Das Logo zum zweiten Mal. \bildquelle{nirgendwo}]{Das Logo zum zweiten Mal.}
\end{figure}
\unsinn
\unsinn
\unsinn
\unsinn
\unsinn
