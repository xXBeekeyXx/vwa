\documentclass[DLS,
	inreferencehack,
	ohneVgl=false,
	ohneS=false,
	scauthor,
	rundeauslassung=false,
	bookstyle=false,
	widowlines=3,
	titlepage=DLS2017,
	listof=nochaptergap,
	doppelpunkt=false,
	postnotedoppelpunkt=false,
	zitierstil=klassisch]{vwa}

\begin{document}
\chapter*{Begleitprotokoll}
\thispagestyle{empty}

\begin{labeling}{Betreuungslehrer\quad}

  \setlength{\itemsep}{0pt}

  \item[Schüler:] {\bfseries Franz Srambical, 8C}

  \item[Thema:] {\bfseries Maschinelle Werteanpassung bei einer hypothetischen allgemeinen künstlichen Intelligenz}

  \item[Betreuungslehrer:] {\bfseries Mag.\ Leonard Michlmayr}
\end{labeling}




\begin{itemize}
  \item 10. 10. 2018: Vorbesprechung
  \item 25. 10. 2018: Besprechung des Zeitungsartikels
  \item 07. 11. 2018: Präzisierung des Themas (Superintelligenz)
  \item 12. 12. 2018: Diskussion über Bücher (Weizenbaum: Die Macht der Computer; Gödel, Escher, Bach)
  \item 09. 01. 2019: Präzisierung des Themas (Allgemeine künstliche Intelligenz statt Superintelligenz)
  \item 16. 01. 2019: Präzisierung des Themas (Maschinelle Werteanpassung bei einer hypothetischen allgemeinen künstlichen Intelligenz)
  \item 22. 01. 2019: Besprechung des Erwartungshorizontes
  \item 23. 01. 2019: Verbesserung des Erwartungshorizontes
  \item 12. 02. 2019: Prof. Sepp Hochreiter und seine Arbeiten + Diskussion zu neuronalen Netzen (Gödelscher Unvollständigkeitssatz)
  \item 26. 02. 2019: Spektrum Spezial (Willkommen in der Welt der Daten)
  \item 19. 03. 2019: Trump und der Staatsstreich der Konzerne (Besprechung der Doku)
  \item 27. 03. 2019: Unsere Daten: Wie berechenbar wir sind (Besprechung der Doku)
  \item 25. 06. 2019: 3Sat-Diskussion über KI
  \item 02. 09. 2019: Anfrage an Hr. Prof. Michlmayr als neuen VWA-Betreuer
  \item 09. 09. 2019: VWA-Gespräch + Themenüberblick
  \item 09. 09. 2019: Mail (VWA -- Einleitung + Kapitel \enquote{Allgemeine künstliche Intelligenz})
  \item 11. 11. 2019: Feedback zu den ersten Kapiteln (bis inkl. \enquote{Probleme einer allgemeinen künstlichen Intelligenz})
  \item 21. 01. 2020: Überarbeitungsvorschläge bezüglich der Einleitung
  \item 31. 01. 2020: Feedback zur VWA exkl. Schluss, Abstract und Kapitel \enquote{Auswirkungen einer AKI}
  \item 09. 02. 2020: Mail (VWA -- Finalfassung)
\end{itemize}
{\centering\vrule height 0.025pt depth 0.025pt width 6em\par}


Die Arbeit hat eine Länge von 43 141 Zeichen.



{\centering\setlength\baselineskip{.5\baselineskip}\vrule height 0.025pt depth 0.025pt width 6em\par} \vskip3cm plus 0.2fil

Wien, 11. Februar 2020\par \vskip2cm {\raggedleft Franz Srambical\par}


%Anzahl der Zeichen: 43 141
\end{document}
%%% Local Variables:
%%% mode: latex
%%% TeX-master: t
%%% End:
