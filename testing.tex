\chapter{Testbeispiele}
\iffalse
\zit[\bibstring{confer}][]{Scherz}{Ein wörtliches Zitat, das mit einem Punkt endet.}
\zit[\bibstring{confer}][]{Scherz}{Ein wörtliches Zitat, das mit einem Fragezeichen endet?}
\zit[\bibstring{confer}][]{Scherz}{Ein wörtliches Zitat, das mit einem Punkt endet}.
\zit[\bibstring{confer}][]{Scherz}{Ein wörtliches Zitat, das mit einem Fragezeichen endet}?
\zit[\bibstring{confer}][]{Scherz}[.]{Ein wörtliches Zitat, das mit einem Punkt endet}
\zit[\bibstring{confer}][]{Scherz}[?]{Ein wörtliches Zitat, das mit einem Fragezeichen endet}

\section{Wörtliche Zitate}
\subsection{Satz endet mit Zitat}
\zit[Im Vorwort zu][]{Scherz}[]{Das haben Sie gesagt}.
\zit[im Vorwort zu][]{Scherz}[?]{Was glauben Sie}
\subsection{Satz wird nach dem Zitat fortgesetzt}
\zit[im Vorwort zu][]{Scherz}[?]{Was glauben Sie}, könnte man dazu fragen.
 
\Textcite[123]{Lessing} erkannte bereits: "`Ein Tisch ist kein gutes Bett."'
\fi

\section{Händisch gesetzte Beispiele}
\subsection{Direkte Zitate}
\textsc{Niemand} (1983, S. 123) erkannte bereits: "`Ein Tisch ist kein
gutes Bett."' Dennoch dauerte es Jahrzehnte, bis diese Erkenntnis in der
Fachwelt gebührende Anerkennung fand. "`Das ist alles nur eine Frage der
Matratze"', entgegnete zum Beispiel \textsc{Autor} (Jahr, S. 12).

\textsc{Niemand}\footnote{\textsc{Niemand}, \textit{Nichts}, S. 123}
erkannte bereits: "`Ein Tisch ist kein gutes Bett."'  Dennoch dauerte es
Jahrzehnte, bis diese Erkenntnis in der Fachwelt gebührende Anerkennung
fand. "`Das ist alles nur eine Frage der Matratze"', entgegnete zum
Beispiel \textsc{Autor}.\footnote{\textsc{Autor}, \textit{Titel}, S. 12}

"`Eine Waschrumpel ist kein Federbett."' (\textsc{Autor}, Jahr, S. 123)
"`Eine Waschrumpel ist kein Federbett."'\footnote{\textsc{Autor},
\textit{Titel}, S. 123}

"`Eine Waschrumpel ist kein Federbett"' (\textsc{Autor}, Jahr, S. 123), könnte man meinen.
"`Eine Waschrumpel ist kein Federbett"'\footnote{\textsc{Autor},
\textit{Titel}, S. 123}, könnte man meinen.

% TODO: Eigentlich besser nur links einziehen.
\begin{addmargin}[1cm]{0pt}
\itshape\strut\hbox to 0pt{\hss"`}Blockzitat mit Anführungszeichen. Eine Waschrumpel
ist kein Federbett. Bekanntlich leidet aber selbst bei einem Federbett
die Schlafqualität erheblich unter einer einzelnen Erbse, die unter der
Matratze platziert wird."' (Autor, Jahr, S. 123)
\end{addmargin}

% Eigentlich besser nur links einziehen.
\begin{addmargin}[1cm]{0pt}
\strut\hbox to 0pt{\hss"`}Blockzitat mit Anführungszeichen. Eine Waschrumpel
ist kein Federbett. Bekanntlich leidet aber selbst bei einem Federbett
die Schlafqualität erheblich unter einer einzelnen Erbse, die unter der
Matratze platziert wird."'\footnote{\textsc{Autor}, \textit{Titel}, S.
123}
\end{addmargin}

\subsection{Indirekte Zitate}
% TODO: es gibt unterschiedliche Ansichten darüber, ob der Fußnotentext
% mit einem Großbuchstaben beginnen soll.
\textsc{Autor} (Jahr, S.\@ 123) meint, dass eine Waschrumpel kein
Federbett sei.
\textsc{Autor}\footnote{Vgl.\@ \textsc{Autor}, \textit{Titel}, S. 123}
meint, dass eine Waschrumpel kein Federbett sei.

Eine Waschrumpel ist bekanntlich kein Federbett. (Vgl.\@ \textsc{Autor},
Jahr, S. 123) Eine Waschrumpel ist bekanntlich kein
Federbett.\footnote{Vgl.\@ \textsc{Autor}, \textit{Titel}, S. 123}

% \KOMAoptions{footnotes=multiple}
Eine Waschrumpel ist zwar kein Federbett (vgl.\@ \textsc{Autor}, Jahr,
S. 123), aber dennoch ist sie ein gutes Musikinstrument. Eine
Waschrumpel ist zwar kein Federbett\footnote{Vgl.\@ \textsc{Autor},
\textit{Titel}, S.
123}\multiplefootnoteseparator\footnote{Experimentelle Fußnote mit ganz
viel Text. So viel Text, dass er nicht in einer Zeile Platz hat. Wegen
der kleinen Schrift in der Fußnote, muss das ganz schön viel Text
sein.},
 aber dennoch ist sie ein gutes Musikinstrument.

\section{Testfeld}
Im folgenden werden die Zitiermakros angewendet und sollten der
eingestellten Zitierweise entsprechend korrekt arbeiten.
\subsection{Direkte Zitate}
\Textcite[123]{Niemand} erkannte bereits: \blockquote{Ein Tisch ist kein
gutes Bett.} Dennoch dauerte es Jahrzehnte, bis diese Erkenntnis in der
Fachwelt gebührende Anerkennung fand. \blockquote{Das ist alles nur eine
Frage der Matratze}, entgegnete zum Beispiel \textcite[12]{Autor}.

\zit[123]{Autor}[.]{Eine Waschrumpel ist kein Federbett}
\zit[123]{Autor}{Eine Waschrumpel ist kein Federbett}, könnte man meinen.
\zit[123]{Autor}[.]{Blockzitat mit Anführungszeichen. Eine Waschrumpel
ist kein Federbett. Bekanntlich leidet aber selbst bei einem Federbett
die Schlafqualität erheblich unter einer einzelnen Erbse, die unter der
Matratze platziert wird}

\subsection{Indirekte Zitate}
% TODO: es gibt unterschiedliche Ansichten darüber, ob der Fußnotentext
% mit einem Großbuchstaben beginnen soll.
\Textcite[\bibstring{confer}][123]{Autor} meint, dass eine Waschrumpel kein
Federbett sei.

Eine Waschrumpel ist bekanntlich kein Federbett. \Vgl[123]{Autor} Eine
Waschrumpel ist zwar kein Federbett \vgl[123]{Autor}{}, aber dennoch ist
sie ein gutes Musikinstrument.
